\chapter*{Resumen}
En este trabajo se desarrolló un sistema automatizado para la calibración de instrumentos de medición de viento, utilizando el túnel de viento del Servicio Meteorológico Nacional. El sistema incluye un sistema de medición y control desarrollado con la placa EDU-CIAA, encargado de la adquisición de datos de viento de los sensores, así como de la comunicación y el control del motor del túnel mediante un controlador de lazo cerrado PID, reemplazando el sistema manual de medición. Además, se implementó una aplicación web utilizando el \textit{framework} \textit{Django} y un servidor \textit{WebSocket} con \textit{Django-Channels} para configurar el sistema de medición y control, cargar información relevante de los sensores y del túnel de viento, presentar las mediciones en tiempo real y procesar los datos mediante un modelo comparativo, con el fin de calcular la incertidumbre y la corrección de las mediciones. Toda la información recopilada se almacena de manera estructurada en una base de datos, permitiendo un acceso eficiente y organizado a los datos. Este desarrollo permitirá al laboratorio del Servicio Meteorológico Nacional calibrar sensores de viento, satisfaciendo así la demanda de la red de estaciones automáticas y los sistemas de anemómetros instalados en el país.

