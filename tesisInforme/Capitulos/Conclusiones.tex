\section{Conclusiones finales}
El trabajo realizado en esta tesis ha permitido una comprensión profunda y detallada de diversos aspectos técnicos y científicos relacionados con la medición y calibración de instrumentos de viento. Desde el estudio del marco teórico de la metrología hasta la familiarización y entendimiento del uso del instrumental, se ha logrado establecer una base sólida para la investigación y desarrollo sobre estos temas. El principal aporte de este trabajo fue la automatización completa del procedimiento de calibración de anemómetros en un túnel de viento. Para lograr esto, se desarrollaron software, hardware y firmware específicos. Un componente clave fue el desarrollo de un \textit{datalogger} para medir el viento con anemómetros y controlar la potencia del túnel de viento, permitiendo la recolección de datos precisos y su análisis en tiempo real, lo que facilitó la calibración de los instrumentos de medición. Además, se implementó una aplicación web con una base de datos y una interfaz de usuario que permite configurar los equipos y cargar los metadatos, proporcionando una plataforma accesible y eficiente para visualizar y gestionar el proceso de calibración. En el proceso de automatización, se aplicaron conocimientos avanzados en programación de sistemas embebidos, electrónica analógica, protocolos de comunicación para redes TCP/IP, diseño de sistemas de control PID y programación web para la interfaz de usuario, y base de datos para el resguardo de la información obtenida, desarrollando así una solución integral y eficiente. Por otro lado, se abordaron los fundamentos del viento, el funcionamiento de los túneles de viento y los instrumentos utilizados para medir el flujo de aire. También, se explicó el cálculo de la incertidumbre, la importancia de los certificados de calibración, los tipos de incertidumbre y la trazabilidad de las mediciones. Estos conocimientos fueron esenciales para garantizar la confiabilidad de los resultados obtenidos. Con el sistema en funcionamiento, se realizaron mediciones y se obtuvieron resultados experimentales que incluyeron la caracterización de la zona de medición y la calibración de un anemómetro. Estos resultados demostraron la efectividad del sistema desarrollado y su capacidad para proporcionar mediciones precisas y confiables. En resumen, este trabajo ha contribuido significativamente al avance del conocimiento en el campo de la metrología del viento, proporcionando herramientas prácticas y automatizadas para mejorar la precisión y eficiencia en la calibración de anemómetros. El sistema automatizado para la calibración de sensores de viento ultrasónicos reduce los tiempos operativos, minimiza el cálculo manual y disminuye los errores sistemáticos, lo que resulta en una mejora sustancial en la calidad de la calibración.

\section{Trabajo futuro}

En cuanto al trabajo futuro, se podrán mejorar las prestaciones del sistema mediante la expansión del hardware y el software para trabajar con otros túneles de viento y otros tipos de anemómetros. También será posible adaptar este sistema a otras variables meteorológicas con las que trabaja el SMN, como presión, temperatura, humedad y radiación, dada la versatilidad y escalabilidad del sistema. Adicionalmente, se elaborará un manual de usuario detallando el procedimiento de uso del sistema implementado. Se actualizará el sistema embebido por uno de mayor disponibilidad en el mercado y se agregará conectividad WiFi para no depender de una conexión Ethernet cableada. Se migrará el firmware generado por software propietario a un desarrollo propio, con el objetivo de no depender de herramientas pagas. Además, se mejorará el diseño del frontend y la funcionalidad, basándose en la experiencia del usuario con este primer prototipo. Por último, se procederá al despliegue en producción de la aplicación web en un servidor dedicado dentro de la infraestructura de servidores locales del SMN, asegurando la migración completa desde la estación de trabajo utilizada durante el desarrollo, con el fin de optimizar la disponibilidad y el rendimiento del sistema.
