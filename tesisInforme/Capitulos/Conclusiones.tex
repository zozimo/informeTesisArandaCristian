\section{Conclusiones finales}
El exhaustivo trabajo realizado en esta tesis ha permitido una comprensión profunda y detallada de diversos aspectos técnicos y científicos relacionados con la medición y calibración del viento. Desde el estudio del marco teórico de la metrología hasta la familiarización y entendimiento del uso del instrumental, se ha logrado establecer una base sólida para la investigación y desarrollo sobre estos temas.

Se abordaron los fundamentos del viento, el funcionamiento de los túneles de viento y los instrumentos utilizados para medir el flujo de aire. Además, se explicó el cálculo de la incertidumbre, la importancia de los certificados de calibración, los tipos de incertidumbre y la trazabilidad de las mediciones. Estos conocimientos fueron esenciales para garantizar la exactitud y confiabilidad de los resultados obtenidos.

La investigación incluyó el desarrollo de software, hardware y firmware necesarios para la aplicación, lo que permitió adquirir conocimientos detallados sobre el procedimiento de calibración de anemómetros en un túnel de viento. Este proceso no solo se comprendió teóricamente, sino que también se aplicó mediante el sistema de calibración automatizado, mejorando así la eficiencia en términos de tiempo y recursos humanos para realizar las mediciones y el procesamiento de datos.

El desarrollo de un datalogger para medir el viento con anemómetros y controlar la potencia del túnel de viento fue un componente clave de este trabajo. Este dispositivo permitió la recolección de datos precisos y su análisis en tiempo real, facilitando la calibración de los instrumentos de medición.

Asimismo, se implementó una aplicación web con una base de datos y una interfaz de usuario que permite configurar los equipos y cargar los metadatos. Esta herramienta proporciona una plataforma accesible y eficiente para visualizar y gestionar el proceso de calibración.

Finalmente, utilizando el sistema desarrollado, se realizaron mediciones y se obtuvieron resultados experimentales que incluyeron la caracterización de la zona de medición y la calibración de un instrumento. Estos resultados demostraron la efectividad del sistema desarrollado y su capacidad para proporcionar mediciones precisas y confiables.

En resumen, este trabajo ha contribuido significativamente al avance del conocimiento en el campo de la metrología del viento y ha proporcionado herramientas prácticas y automatizadas para mejorar la precisión y eficiencia en la calibración de anemómetros.

\section{Trabajo futuro}

En cuanto al trabajo futuro, se podrá mejorar las prestaciones del sistema, expandiendo el hardware y el software para trabajar con otro tipo de sensores de viento. Ademas se podrá expandir su aplicación a otras variables meteorológicas con las que se trabaja en el SMN, como presión, temperatura, humedad y radiación dada la versatilidad y escalabilidad del hardware y software. También sera posible adaptar este sistema en otros túneles de viento.



